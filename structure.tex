% Dimensione del font
\usepackage[fontsize=11pt]{scrextend}
% Lingua del testo
\usepackage[italian, english]{babel}
\usepackage[english=usenglishmax]{hyphsubst}
\usepackage[utf8]{inputenc} 
\usepackage[T1]{fontenc}
% Per modificare l'header delle pagine 
\usepackage{fancyhdr}

% Librerie matematiche
\usepackage{amssymb}
\usepackage{amsmath}
\usepackage{amsthm}   
\usepackage{mathtools}
% Uso delle immagini
\usepackage{graphicx}
%scritta Figura n in grassetto
\usepackage[labelfont=sf]{caption,subcaption}
%stile delle citazioni
\usepackage[square,numbers,sort&compress]{natbib}
%tabelle
\usepackage{dcolumn}
\usepackage{multicol}
    \newcommand{\mcc}[1]{\multicolumn{1}{c}{#1}}
\usepackage{booktabs}
\usepackage{float}
\usepackage{longtable}
\usepackage{rotating} 
\usepackage{lscape}
\usepackage{caption}
\usepackage{longtable}
\usepackage{blindtext}
\usepackage{tabularx}

\usepackage{afterpage}

\usepackage{ragged2e} 
\usepackage{changepage}

\usepackage{epigraph} 

%stile dei capitoli
\usepackage[Sonny]{fncychap}[l]

% Per inserire gli hyperlink tra i vari elementi del testo 
\usepackage{hyperref}  
\hypersetup{
    colorlinks=true,
    linkcolor=black,
    citecolor=black,
    }
% Rimuove il numero di pagina all'inizio dei capitoli
\fancypagestyle{plain}{
  \fancyfoot[OR, EL]{\sffamily\thepage}
  \fancyhead{}
  \renewcommand{\headrulewidth}{0pt}
}

% Togliendo il commento al comando che segue, si inseriscono nella bibliografia anche le fonti presenti in Bibliography.bib ma non citati direttamente con il comando \cite

% Aggiunti definizioni, teoremi, linea e listing
\theoremstyle{definition} % Define theorem styles here based on the definition style (used for definitions and examples)


%\theoremstyle{plain} % Define theorem styles here based on the plain style (used for theorems, lemmas, propositions)
\newtheorem{theorem}{Theorem}[section]
\newtheorem{definition}[theorem]{Definition}
\newtheorem{corollary}[theorem]{Corollary}
\newtheorem{prop}[theorem]{Proposition}
\newtheorem{lemma}[theorem]{Lemma}
\newtheorem{postulate}{Postulate}
\newtheorem*{postulate*}{Postulate}

\theoremstyle{remark} % Define theorem styles here based on the remark style (used for remarks and notes)
\newtheorem{remark}[theorem]{Remark}
\newtheorem{ex}[theorem]{Example}

\newenvironment{soln}{\begin{proof}[Soluzione]}{\end{proof}}

\numberwithin{equation}{section} 

%\raggedbottom

\usepackage[dvipsnames]{xcolor}
\definecolor{blu_unipi}{RGB}{0, 81, 139}

%per aggiungere commenti
\newcommand\myworries[1]{\textcolor{red}{#1}}

%stile dei titoli
\usepackage{titlesec}
\titleformat*{\section}{\Large\sffamily}
\titleformat*{\subsection}{\large\sffamily}
\titleformat*{\subsubsection}{\itshape\subsubsectionfont}
\usepackage{braket}

%\renewcommand*\familydefault{\sfdefault}

%

\setlength{\headheight}{13.59999pt}

\usepackage{relsize}

%cose generiche

\newcommand{\R}[0]{\mathbb{R}}
\newcommand{\C}[0]{\mathbb{C}}
\newcommand{\Hi}[0]{\mathbb{H}}
\newcommand{\N}[0]{\mathbb{N}}
\newcommand{\Q}[0]{\mathbb{Q}}
\newcommand{\Fi}[0]{\mathbb{F}}
\newcommand{\Pa}[0]{\mathcal{P}}
\newcommand{\U}[0]{\mathcal{U}}
\newcommand{\Fin}[0]{\text{Fin}}
\newcommand{\inv}[1][f]{#1^{-1}}
\newcommand{\calpha}[1][\R^{d\times d}]{\mathcal{C}^\alpha([0,T],#1)}
\newcommand{\co}[0]{\mathbf{c}}
\newcommand{\ct}[0]{\mathbf{\Tilde{c}}}
\newcommand{\qubs}[0]{\vphantom{\mathbb{H}}^\text{\textparagraph}\mathbb{H}}


\chapter*{Introduction}

\addcontentsline{toc}{chapter}{Introduction}

The Sewing Lemma is a powerful tool in the study of rough paths and differential equations, widely known for its role in constructing generalized integrals in cases where classical techniques are insufficient due to low regularity. 

While the results presented in this thesis are well established in the literature, they do not appear to have been applied in the specific context of quantum computing before. This thesis explores such an application, focusing on the extension of the Sewing Lemma to the non-commutative setting, which is essential for dealing with problems in quantum mechanics, where matrix-valued and operator-valued functions dominate.


In particular, the non-commutative extension developed here offers new tools for addressing irregular Hamiltonians. While much of the existing theory assumes smooth or at least differentiable operators, our approach allows for a less regular setting, broadening the scope of applicable models in quantum mechanics. 

A significant part of this thesis is dedicated to outlining the technical aspects of this extension and discussing its potential for advancing (growing) fields such as quantum computing. 

Although quantum machine learning will not be treated here, it is worth mentioning that this rapidly growing field will likely benefit from the mathematical techniques introduced in this work, especially as quantum algorithms increasingly deal with low-regularity operators.
